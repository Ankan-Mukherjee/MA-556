
\begin{frame}
\frametitle{Problem 2}
\begin{block}{Problem 2 of Notes Chapter 2}
For any field $\mathbb{K}$ and $\mathbb{K}$-algebra $A$, we have an inclusion $\mathbb{K}\xhookrightarrow{} A$. Show that:\\
For any derivation $D$ of $A$, we have $D(c) = 0$ for any scalar $c\in\mathbb{K}$. Is the
same true for a derivation $D$ of $A$ into a bimodule $M$?
\end{block}
\end{frame}

\begin{frame}
\frametitle{Problem 2 (Contd.)}
Recall the definition of a derivation on an algebra $A$.
\begin{block}{Definition of Derivation}
For any $\mathbb{K}$-algebra $A$, and a bimodule $M$, a linear map $D:A\rightarrow M$ is a derivation iff\\
\begin{enumerate}
\item $D(a+b)=D(a)+D(b)$
\item $D(ca)=cD(a)$ for all scalars $c$
\item $D(ab)=a.D(b)+D(a).b$
\end{enumerate}
\end{block}
\end{frame}


\begin{frame}
\frametitle{Problem 2 (Contd.)}
In a field $\mathbb{K}$ and $\mathbb{K}$-algebra $A$, we attempt to calculate $D(c)$ for $c\in\mathbb{K}$, $c$ being a scalar. Our inclusion map is $\mathbb{K}\xhookrightarrow{} A$.
\begin{align*}
D(c) &= D(c\times\mathbb{I})&&(\text{property of }\mathbb{I})\\
&= cD(\mathbb{I})&&(\text{definition of scalar})\\
&= cD(\mathbb{I}\times\mathbb{I})&&(\mathbb{I}\times\mathbb{I}=\mathbb{I})\\
&= c(D(\mathbb{I})\times\mathbb{I}+\mathbb{I}\times D(\mathbb{I}))&&(\text{Leibniz Rule})\\
&= c(D(\mathbb{I})+D(\mathbb{I}))&&(\text{property of }\mathbb{I})\\
&= 2cD(\mathbb{I})
\end{align*}
Thus,
\begin{align*}
&D(c) = cD(\mathbb{I})= 2cD(\mathbb{I})\\
\Rightarrow &D(c)=0
\end{align*}
A similar argument works on bimodules. Derivation of any scalar will be $0$.
\end{frame}
